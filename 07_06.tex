% \documentclass[12pt]{article}
\usepackage{enumitem}
\usepackage{amsmath}
\usepackage{amssymb}
\newcommand{\prompt}[1]{\textbf{#1}\\}
\newcommand{\original}[1]{
    \textbf{Original}\\
        {#1}\\
}
\newcommand{\premise}[1]{\textsc{Premise\\ }#1\\}
\newcommand{\conclusion}[1]{\textsc{Conclusion\\ }#1\\}
\newcommand{\premandconc}[1]{\textbf{Premise and Conclusion\\}#1\\}

\newcommand{\exercise}[1]{\textbf{Premise and Conclusion\\}#1\\}

\renewcommand\symbol[2]{\item $#1$: #2}
\newcommand\symbolization[1]{\\$\mathbf{#1}$}
\newenvironment{statement}[1]{
    \item #1
        \begin{itemize} [label=, topsep=0pt]
}{
    \end{itemize}
}

\usepackage{amssymb}

\author{Idris Raja}

% \date{2023/09/27}
% \begin{document}
% \maketitle

\section{\S 7.6 Conditional Arguments}
\begin{enumerate}
	\begin{statement}{Provided the fetus is a person, a fetus has a right to life. Should a fetus
			have a right to life, it is false that someone has the right to take its
			life. However, if abortions are moral, someone does have the right to
			take the life of a fetus. Consequently, if a fetus is a person, abortions
			are not moral.
		}
		\symbol{A}{The fetus is a person}
		\symbol{B}{fetus has a right to life}
		\symbol{C}{Someone has a right to take the life of a fetus}
		\symbol{D}{Abortions are moral}
		\symbolization{A \implies B, B \implies \neg C, D \implies C,  \therefore A
			\implies \neg D}
	\end{statement}

	\begin{statement}{Lung cancer is not caused by smoking, and this is so for the
			following reasons. Lung cancer is more common among male smokers than among
			female smokers. If smoking were the cause of lung cancer, this would not be
			true. The fact that lung cancer is more common among male smokers implies
			that it is caused by something in the male makeup. But if it is caused by
			this, it is not caused by smoking. }
		\symbol{A}{Lung cancer is caused by smoking}
		\symbol{B}{Lung cancer is more common among male smokers than female smokers}
		\symbol{C}{Lung cancer is caused by something in the male makeup}
		\symbolization{B, A \then \neg B, B \then C, C \then \neg A, \therefore \neg A}
	\end{statement}

	\begin{statement}
		{Some people do not think that the abortion issue turns solely on the
			question of whether the fetus is a person. They would offer the following
			argument: Abortion isn't wrong, since it is wrong only if a fetus has the
			right to use the organs of other persons to stay alive. But (since no one
			has that right) a fetus doesn't have that right either.}
		\symbol{W}{Abortion is wrong}
		\symbol{U}{A fetus has a right to use the organs of another person to stay alive}
		\symbolization{W \then U, \neg U \therefore \neg W}
	\end{statement}

	\begin{statement}
		{If the ventricles transmit two ounces of blood at each beat, then if they
			beat 60 times a minute, then they will transmit over seven pounds of blood
			in a minute. Let us suppose that both these conditions are true. They won't
			transmit seven pounds of blood in a minute unless the veins supply them with
			that much blood. If the blood doesn't circulate, there is no way the veins
			can supply that much blood. From this it is clear that the blood circulates.
		}
		\symbol{V}{Ventricles transmit two ounces of blood at each beat;}
		\symbol{B}{The heart beats 60 times per minute;}
		\symbol{T}{The ventricles transmit seven pounds of blood per minute;}
		\symbol{W}{The veins supply the ventricles with seven pounds of blood per minute;}
		\symbol{C}{Blood circulates}
		\symbolization{V \then (B \then T), V, B, \neg W \then \neg T, \neg C \then
			\neg W \therefore C}
	\end{statement}

	\begin{statement}{If intelligence is wholly hereditary and identical twins have
			the same heredity, then being raised in separate households will not reduce
			the similarity of intelligence between identical twins, but it does reduce
			the similarity. Identical twins have come from a common sperm and egg. This
			last is so if, and only if, the twins have identical heredity. Therefore,
			intelligence is not entirely hereditary.

			(Ask yourself whether if you were to assert the first premise, you would also be
			asserting that being raised in separate households does reduce the similarity
			between identical twins. If your intuition is yes, then the first premise must
			be a conjunction.)}

		\symbol{H}{Intelligence is entirely hereditary}
		\symbol{I}{Being raised in separate households will reduce similarity of intelligence between identical twins}
		\symbol{T}{Identical twins have the same heredity}
		\symbol{P}{Identical twins come from a common sperm and egg}
		\symbolization{((H \land T) \then \neg I) \land I, P, P \iff T, \therefore \neg H}
	\end{statement}

	\begin{statement}{If there are an infinite number of points in a finite line L,
			then if those points have size, L will be infinitely long, and if they do
			not have size, L won't have length. Line L is neither infinitely long nor
			without length. This proves that there are not an infinite number of points
			in L.}
		\symbol{I}{There are an infinite number of points in a finite line}
		\symbol{H}{Points have size}
		\symbol{L}{Line L is infinitely long}
		\symbol{G}{Line L has length}
		\symbolization{I \then ((H \then L) \land (\neg H \then \neg G)), \neg L
			\land G, \therefore \neg I}
	\end{statement}

	\begin{statement}{ The following argument is sometimes called ‘Pascal's wager',
			named after Blaise Pascal, a seventeenth-century French philosopher and
			mathematician: If I believe in God, then if He exists, I gain, and if He
			doesn't, then I don't lose. If, on the other hand, I don't believe in God,
			then if He exists, I lose, and if He doesn't, I don't gain. From this it
			follows that if I believe, I'll either gain or not lose, while if I don't
			believe, I'll either lose or fail to gain.}
		\symbol{B}{I believe in God}
		\symbol{E}{God exists}
		\symbol{G}{I gain}
		\symbol{L}{I lose}
		\symbolization{(B \then ((E \then G) \lor (\neg E \then \neg L))) \lor (\neg B \then ((E \then L) \lor (\neg E \then \neg G)))}
		\symbolization{\therefore (B \then (G \lor \neg L)) \land (\neg B \then (L
			\lor \neg G))}
	\end{statement}

	\begin{statement} {If the world is disordered, it cannot be reformed unless a
			sage appears, but no sage can appear if the world is disordered. It follows
			that the world cannot be reformed if it is disordered.}
		\symbol{D}{The world is disordered}
		\symbol{W}{The world can be reformed}
		\symbol{S}{A sage can appear}
		\symbolization{D \then (\neg S \then \neg W), D \then \neg S, \therefore D \then \neg W}
	\end{statement}

	\begin{statement} { People won't be given bribes if we discourage bribery. But
			unless people are given bribes, cases of bribery won't come before the
			public attention, and people won't know that bribery is wrong unless it
			comes to their attention. Given all this and the moral principle: bribery is
			wrong provided that people know that it is wrong, we can draw the following
			startling conclusion: Bribery is wrong only if we don't discourage it. }
		\symbol{B}{Bribery is wrong}
		\symbol{D}{We discourage bribery}
		\symbol{G}{People will be given bribes}
		\symbol{A}{Cases of bribery will come before the public attention}
		\symbol{K}{People know that bribery is wrong}
		\symbolization{D \then \neg G, (\neg G \then \neg A) \land (\neg A \then
			\neg K), K \then B \therefore B \then \neg D}
	\end{statement}

	\begin{statement} {George will leave unless Mary doesn't leave. But unless Phil
			stays, she will leave, and George won't leave provided that it rains. As a
			result, if it rains, Mary won't leave. }
		\symbol{G}{George will leave}
		\symbol{M}{Mary does leave}
		\symbol{P}{Phil stays}
		\symbol{R}{It rains}
		\symbolization{M \then G, (\neg P \then M) \land (R \then \neg G), \therefore
			R \then \neg M)}
	\end{statement}

	\begin{statement} { All high-ranking officers in our present army are
			volunteers. Military coups are engineered by high-ranking officers. Provided
			that both the preceding statements are true, an all-volunteer army would not
			increase the chances of a military takeover. Consequently, the chances of a
			military coup would not be increased by the institution of an all-volunteer
			army.}
		\symbol{H}{All high-ranking officers in our present army are volunteers}
		\symbol{E}{Military coups are engineered by high-ranking officers}
		\symbol{I}{an all-volunteer army will increase the chances of a military takeover}
		\symbolization{H, E, (H \land E) \then \neg I \therefore \neg I }
	\end{statement}

	\begin{statement} {It is not the case that there will be both a continuation of PLO raids in Israel
			and peace in the Middle East. The PLO raids on Israel will not continue only if
			Arabs gain autonomy on the West Bank. So there will be no peace in the Middle
			East, since the West Bank Arabs will not gain autonomy.}
		\symbol{C}{There will be a continuation of PLO raids in Israel}
		\symbol{P}{There will be peace in the Middle East}
		\symbol{A}{Arabs gain autonomy on the West Bank}
		\symbolization{\neg (C \land P), \neg C \then A \therefore \neg A \then \neg P}
	\end{statement}

	\begin{statement} {Hopes for peace in the Middle East are unrealistic unless Clinton makes a secret
			deal with the Israeli prime minister. Clinton will make a secret deal with the
			Israeli prime minister if and only if he is not as inexperienced in foreign
			affairs as we have been led to believe. Since Clinton is as inexperienced in
			foreign affairs as we have been led to believe, hopes for peace in the Middle
			East are unrealistic.}
		\symbol{H}{Hopes for peace in the Middle East are realistic}
		\symbol{C}{Clinton makes a secret deal with the Israeli prime minister}
		\symbol{E}{Clinton is as experienced in foreign affairs as we have been led to believe}
		\symbolization{\neg C \then \neg H, C \iff E, \neg E \therefore \neg H}
	\end{statement}

	\begin{statement} {We will play tennis and go jogging only if the temperature
			reaches 45 degrees. It will not reach 45 degrees. So, we will not play
			tennis, and we will not go jogging.}
		\symbol{T}{We play tennis}
		\symbol{J}{We jog}
		\symbol{R}{The temperature reached 45 degrees}
		\symbolization{(T \land J) \then R, \neg R, \therefore \neg T \land \neg J}
	\end{statement}

	\begin{statement} {The Vietnamese ceasefire will lead to a permanent peace only
			if neither the Communists nor the Saigon regime retain their former tactics.
			But they are not going to give up their former tactics. So the ceasefire
			will not lead to a permanent peace.}
		\symbol{P}{The Vietnamese ceasefire will lead to permanent peace}
		\symbol{C}{Communists retain their tactics}
		\symbol{S}{Saigon retains its tactics}
		\symbolization{P \then (\neg C \land \neg S), C \land S, \therefore \neg P}
	\end{statement}

	% The problem in each of the following passages is to determine whether or not
	% it contains an argument, and, if so, to reconstruct it. Reconstruction involves
	% distinguishing between premises and conclusion, and, perhaps, supplying miss-
	% ing premises and/or conclusion. Once you have reconstructed an argument,
	% symbolize it in PL.

	\begin{statement}{ ‘What is not so generally recognized is that there can be no
			way of proving that the existence of a god, such as the God of Christianity,
			is even probable. Yet this also is easily shown. For if the existence of
			such a god were probable, then the proposition that he existed would be an
			empirical hypothesis. And in that case it would be possible to deduce from
			it, and other empirical hypotheses, certain experiential propositions which
			were not deducible from these other hypotheses alone. But in fact this is
			not possible' (A. J. Ayer, Language, Truth, and Logic).}

		\symbol{G}{There is a way of proving that the existence of God is probable}
		\symbol{E}{The proposition that God exists is an empirical hypothesis}
		\symbol{C}{It is possible to deduce from the proposition that God exists certain experiential propositions}
		\symbolization{G \then E, E \then C, \neg C \therefore \neg G}
	\end{statement}

	\begin{statement}{‘As a challenge to theism, the problem of evil has
			traditionally been posed in the form of a dilemma: if God is perfectly
			loving, he must wish to abolish evil; and if he is all-powerful, he must be
			able to abolish evil. But evil exists; therefore, God cannot be both
			omnipotent and perfectly loving' (missing premise) (John Hick, Philosophy of
			Religion).}
		\symbol{W}{God wants to abolish evil}
		\symbol{A}{God is able to abolish evil}
		\symbol{E}{Evil exists}
		\symbol{L}{God is all loving}
		\symbol{P}{God is all-powerful}
		\symbol{M}{God wishes to abolish evil}
		\symbolization{(W \land A) \then \neg E, (L \then W) \land (P \then A), E, \therefore \neg (P \land L)}
	\end{statement}

	\begin{statement} {Background information: suppose you are on the committee
			which has to plan the release of tickets for a big rock concert; you are
			faced with a problem. The problem is that some of your helpers will be out
			of town next week, and the concert is coming up soon; so, if you do not give
			out the tickets until later, there will not be enough lead time before the
			concert. You are then presented with the following valid argument by one ofj
			your helpers: If we have the ticket release next week, we will be
			short-handed for the release. But if we have the release later, people won't
			be happy with us. But in fact, people will be happy with us only if we are
			not short-handed for the release (because they do not want to have to wait
			too long in line). So, people will not be happy with us.}
		\symbol{R}{The ticket release is next week}
		\symbol{S}{We are short-handed on ticket-sellers}
		\symbol{H}{People will be happy}
		\symbolization{(R \then S), (\neg R \then \neg H), H \then \neg S,
			\therefore \neg H}
	\end{statement}

	\begin{statement}{There will be a revolution unless something is done to improve
			living conditions among the poorer classes. But nothing will be done to
			improve those conditions. Hence, there will be a revolution.}
		\symbol{D}{Something is done to improve living conditions among the poorer classes}
		\symbol{R}{There will be a revolution}
		\symbolization{\neg D \then R, \neg D \therefore R }
	\end{statement}

	\begin{statement}{If Frank and Bill ride bikes, then Mary doesn't ride a bike
			only if John rides one. Bill doesn't ride a bike unless John rides one.
			Therefore, Mary rides a bike if, and only if, John rides a bike.}
		\symbol{F}{Frank rides a bike}
		\symbol{B}{Bill rides a bike}
		\symbol{M}{Mary rides a bike}
		\symbol{J}{John rides a bike}
		\symbolization{(F \land B) \then (\neg M \then J), \neg J \then \neg B \therefore M \iff J}
	\end{statement}

	\begin{statement}{ If Harry takes drugs, then if he drinks, he is going to get
			himself into some serious trouble. But Bill drinks without Harry drinking.
			And Frank doesn't take drugs if Harry takes drugs. Therefore, Bill drinks
			and Frank takes drugs only if Harry drinks and takes drugs.}
		\symbol{T}{Harry takes drugs}
		\symbol{D}{Harry drinks}
		\symbol{S}{Harry is going to get himself into some serious trouble}
		\symbol{B}{Bill drinks}
		\symbol{F}{Frank takes drugs}
		\symbolization{T \then (D \then S), B \land \neg D, T \then \neg F \therefore (B \land F) \then (D \land T)}
	\end{statement}

	\begin{statement}{Two people are on the train provided that bill got off. bill's
			getting off is necessary for his getting to work on time. therefore, if bill
			got to work on time, there are two people on the train.}
		\symbol{B}{Bill got off the train}
		\symbol{T}{Two people are on the train}
		\symbol{W}{Bill got to work on time}
		\symbolization{B \then T, W \then B, \therefore W \then T}
	\end{statement}

	\begin{statement}{If cars have engines and provided that trains have engines,
			then boats have engines, but boats don't have engines. Therefore, cars have
			engines just in case trains have engines.}
		\symbol{C}{Cars have engines}
		\symbol{T}{Trains have engines}
		\symbol{B}{Boats have engines}
		\symbolization{(C \land T) \then B, \neg B, \therefore C \iff T}
	\end{statement}

	\begin{statement}{‘If George were leaving, I think he'd tell me first,' Mr
			Rogers said. ‘He hasn't told me. He hasn't left.'}
		\symbol{G}{George leaves}
		\symbol{T}{George tells Mr Rogers first}
		\symbolization{G \then T \land \neg T \therefore \neg G}
	\end{statement}

	\begin{statement}{At this very point a disturbing, though obvious, question
			intrudes. If Harry's thesis is false, then there is no point in his having
			written the book or our reading it. But if his thesis is true, then there is
			also no point in his having written the book or our reading it.}
		\symbol{T}{Harry's thesis is true}
		\symbol{W}{There is some point in Harry's having written the book}
		\symbol{R}{There is some point in our having read the book}
		\symbolization{(\neg T \then (\neg R \land \neg W)) \land (T \then (W \land
			R)) \therefore \neg R \land \neg W}
	\end{statement}

	\begin{statement}{Some later Greek writer used erosion by water as evidence for
			the temporal origin of the earth; for, they argued, if the earth has
			existed from eternity, all mountains and other features would by now have
			disappeared.}
		\symbol{E}{The earth had a beginning}
		\symbol{M}{The mountains still exist}
		\symbol{N}{mountains and other features are created}
		\symbolization{\neg E \then (M \land \neg N), \neg\neg M \therefore E}
	\end{statement}
\end{enumerate}

% \end{document}
